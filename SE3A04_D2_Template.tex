\documentclass[]{article}

% Imported Packages
%------------------------------------------------------------------------------
\usepackage{amssymb}
\usepackage{amstext}
\usepackage{amsthm}
\usepackage{amsmath}
\usepackage{enumerate}
\usepackage{fancyhdr}
\usepackage[margin=1in]{geometry}
\usepackage{graphicx}
\usepackage{array}
%\usepackage{extarrows}
%\usepackage{setspace}
%------------------------------------------------------------------------------

% Header and Footer
%------------------------------------------------------------------------------
\pagestyle{plain}  
\renewcommand\headrulewidth{0.4pt}                                      
\renewcommand\footrulewidth{0.4pt}                                    
%------------------------------------------------------------------------------

% Title Details
%------------------------------------------------------------------------------
\title{Deliverable \#2 Template}
\author{SE 3A04: Software Design II -- Large System Design}
\date{}                               
%------------------------------------------------------------------------------

% Document
%------------------------------------------------------------------------------
\begin{document}

\maketitle	
\noindent{\bf Tutorial Number:} T0x\\
{\bf Group Number:} Gx \\
{\bf Group Members:} 
\begin{itemize}
	\item List all Group Member Names (as listed in Avenue)
	\item You do not need to use student \#s or macid (keep those private).
\end{itemize}

\section*{IMPORTANT NOTES}
\begin{itemize}
	%	\item You do \underline{NOT} need to provide a text explanation of each diagram; the diagram should speak for itself
	\item Please document any non-standard notations that you may have used
	\begin{itemize}
		\item \emph{Rule of Thumb}: if you feel there is any doubt surrounding the meaning of your notations, document them
	\end{itemize}
	\item Some diagrams may be difficult to fit into one page
	\begin{itemize}
		\item Ensure that the text is readable when printed, or when viewed at 100\% on a regular laptop-sized screen.
		\item If you need to break a diagram onto multiple pages, please adopt a system of doing so and thoroughly explain how it can be reconnected from one page to the next; if you are unsure about this, please ask about it
	\end{itemize}
	\item Please submit the latest version of Deliverable 1 with Deliverable 2
	\begin{itemize}
		\item Indicate any changes you made.
	\end{itemize}
	\item If you do \underline{NOT} have a Division of Labour sheet, your deliverable will \underline{NOT} be marked
\end{itemize}

\newpage
\section{Introduction}
\label{sec:introduction}
% Begin Section


\subsection{Purpose}
\label{sub:purpose}
% Begin SubSection
This document provides a description of the architectural design decisions of the Secure Chat internal messaging application. Included are analysis class diagrams for the messaging and database systems that make up the functionality of the application. 

This document is intended for internal stakeholders, including but not limited to, project managers, developers, domain experts, and team members/investors. Deliverable 1 should be read prior, and technical knowledge may be beneficial in better understanding the contents of the document. 
% End SubSection

\subsection{System Description}
\label{sub:system_description}
% Begin SubSection
Secure Chat is an Android-based internal messaging application made for workplace communication on company-issued devices, operates on a client-server architecture for efficient division into client and server components, enabling scalability. Additionally, it adopts a repository-style architecture, ensuring the distinct separation of database logic from the overall application structure.
% End SubSection

\subsection{Overview}
\label{sub:overview}
% Begin SubSection
\indent \par
Section 1: Introduction and Context
This section serves as an introductory overview, providing essential context for the subsequent sections of the document.

Section 2: Analysis Class Diagram
In Section 2, an in-depth analysis class diagram is presented, offering a visual representation of the various classes within the application and the intricate relationships existing among them.

Section 3: Architectural Design
Section 3 delves into the architectural design chosen for our system, elucidating the strategic division of subsystems to ensure a cohesive and efficient structure.

Section 4: Class Responsibilities - CRC Cards
Section 4 systematically assigns CRC (Class-Responsibility-Collaboration) cards to each class identified in the analysis class diagram. These cards delineate the responsibilities of each class and highlight their interdependencies with other classes.

Section A: Division of Labour
Concluding the document, Section A provides a comprehensive Division of Labour, outlining the individual contributions of each team member to the creation of this document.
% End SubSection

% End Section

\section{Analysis Class Diagram}
\label{sec:analysis_class_diagram}
% Begin Section
This section should provide an analysis class diagram for your application.
% End Section


\section{Architectural Design}
\label{sec:architectural_design}
% Begin Section
This section will provide an overview of our architecture and the reason we chose it.
\subsection{System Architecture}
\label{sub:system_architecture}
The SCAA uses a Data Centred Software Architecture, mainly the Repository Architecture style. The Repo is where all of the data is stored and is connected to different management subsystems which are all invoked through the user. 

\begin{table}[h]
	\centering
	\begin{tabular}{|>{\centering\arraybackslash}m{4cm}|>{\centering\arraybackslash}m{7cm}|>{\centering\arraybackslash}m{4cm}|}
	\hline
	\textbf{Subsystem} & \textbf{Purpose} & \textbf{Architecture Style} \\ \hline
	Chat Manager & Create, manage and secure chats & Repository and Pipes and Filters \\ \hline
	Account Manager & Create, edit, and delete accounts & Repository \\ \hline
	Announcement Manager & Create, secure and manage announcements & Repository \\ \hline
	\end{tabular}
	\caption{Subsystem Architecture Styles}
	\label{table:subsystem}
\end{table}

In addition four databases are present in the Repo part of the architecture, a chat database, account database, announcement database, and a KDC database. These are to ensure that everyone’s respective information and all the chat information is stored.
\medskip

For this project, we chose the repository architecture due to its perfect fit for our needs in creating a data-centred, secure chat application. This architecture style is the best option as it supports active user engagement with a more structured and passive data management approach, which is crucial for maintaining the integrity and security of communication. The repository model offers a centralised data management system that facilitates efficient data access, sharing, and updating across the application, ensuring that all interactions are consistently secure and coherent.
\medskip

Moreover, this architecture allows for a scalable and modular design, enabling easy integration of additional features or updates without disrupting the existing system. It supports concurrent access and modification of data, a critical requirement for a real-time chat application where multiple users interact simultaneously. The centralised control over data also simplifies the implementation of comprehensive security measures, including the management of encryption keys and authentication protocols, further aligning with our project's stringent security requirements.
\medskip

Additionally, for the specific task of decryption and encryption within our system, we've decided to integrate a pipe and filter architecture into its own subsystem (the Chat Management subsystem). This decision stems from the architecture's proven effectiveness in processing streams of data in real-time, offering a flexible and efficient way to handle the decryption process. This modular approach allows for seamless data flow and transformation, ensuring that decrypted messages are promptly and securely delivered, enhancing the overall performance and reliability of our secure communication solution.
\medskip

As for pipes and filters we didn’t go with this option since there are many limitations to using this for a secure chat app.The pipes and filters architecture is used for processing data through a series of transformations (filters), each performing a specific operation, connected by pipes (data channels). Pipes and Filters focus on data processing rather than centralised data management, potentially complicating the enforcement of security policies across multiple stages. Real-time, stateful interactions required for a chat system are not the primary focus of this architecture, making it less suitable as the core architecture for the application.
\medskip

As for batch sequential it also doesn’t fit the criteria for this project. The way batch sequential works is it takes data in batches through a sequence of steps, each step completing before the next begins. As for this app it requires real-time communication and demands immediate processing, which doesn’t align well with the batch processing nature of this architecture. Adapting batch sequential processing to the dynamic, interactive requirements of a secure chat application can be challenging.


\subsection{Subsystems}
\label{sub:subsystems}
% Begin SubSection
 Provide a list of your subsystems, with a brief description of each. Be sure to document its purpose and relationship to other subsystems.

% End SubSection

% End Section
	
\section{Class Responsibility Collaboration (CRC) Cards}
\label{sec:class_responsibility_collaboration_crc_cards}
% Begin Section
This section should contain all of your CRC cards.

\begin{itemize}
	\item Provide a CRC Card for each identified class
	\item Please use the format outlined in tutorial, i.e., 
	\begin{table}[ht]
		\centering
		\begin{tabular}{|p{5cm}|p{5cm}|}
		\hline 
		 \multicolumn{2}{|l|}{\textbf{Class Name:}} \\
		\hline
		\textbf{Responsibility:} & \textbf{Collaborators:} \\
		\hline
		\vspace{1in} & \\
		\hline
		\end{tabular}
	\end{table}
	
\end{itemize}
% End Section

\appendix
\section{Division of Labour}
\label{sec:division_of_labour}
% Begin Section
Include a Division of Labour sheet which indicates the contributions of each team member. This sheet must be signed by all team members.
% End Section


\end{document}
%------------------------------------------------------------------------------